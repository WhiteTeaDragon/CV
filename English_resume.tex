\documentclass[a4paper,10pt]{article}

%A Few Useful Packages
\usepackage[russian, english]{babel}
\usepackage[utf8]{inputenc}
\usepackage[T2A]{fontenc}
\usepackage{marvosym}
%\usepackage{fontspec} 					%for loading fonts
\usepackage{xunicode,xltxtra,url,parskip} 	%other packages for formatting
\RequirePackage{color,graphicx}
%\usepackage[usenames,dvipsnames]{xcolor}
\usepackage[big]{layaureo} 				%better formatting of the A4 page
% an alternative to Layaureo can be ** \usepackage{fullpage} **
\usepackage{supertabular} 				%for Grades
\usepackage{titlesec}					%custom \section
\usepackage{indentfirst}
\usepackage{amsmath}
\usepackage{amssymb}
\usepackage{enumerate}
\usepackage[cm]{fullpage}
\usepackage{pgfplots}
\usepackage{array}
\usepackage{tabularx}
\usepackage{tikz}
\usepackage{multirow}
\usepackage{longtable}
\setlength{\textheight}{265mm}

\newcommand{\grade}[1]{%
\begin{tikzpicture}
\clip (1em-.3em,-.3em) rectangle (5em +.5em ,.3em);
\begin{scope}
\clip (1em-.3em,-.3em) rectangle (#1em +.5em ,.3em);
\foreach \x in {1,2,...,5}{
 \path[fill=black] (\x em,0) circle (.25em);
}
\end{scope}
\foreach \x in {1,2,...,5}{
 \draw (\x em,0) circle (.25em);
}
\end{tikzpicture}%
}

%Setup hyperref package, and colours for links
\usepackage{hyperref}
\definecolor{linkcolour}{rgb}{0,0.2,0.6}
\hypersetup{colorlinks,breaklinks,urlcolor=linkcolour, linkcolor=linkcolour}

%FONTS
\defaultfontfeatures{Mapping=tex-text}
%\setmainfont[SmallCapsFont = Fontin SmallCaps]{Fontin}
%%% modified for Karol Kozioł for ShareLaTeX use
\setmainfont[
SmallCapsFont = Fontin-Italic.otf,
BoldFont = Fontin-Bold.otf,
ItalicFont = Fontin-Italic.otf
]
{Fontin-Regular.otf}
%%%

%CV Sections inspired by: 
%http://stefano.italians.nl/archives/26
\titleformat{\section}{\Large}{}{0em}{}[\titlerule]
\titlespacing{\section}{0pt}{3pt}{3pt}
%Tweak a bit the top margin
%\addtolength{\voffset}{-1.3cm}

%Italian hyphenation for the word: ''corporations''
\hyphenation{im-pre-se}

%-------------WATERMARK TEST [**not part of a CV**]---------------
\usepackage[absolute]{textpos}

\setlength{\TPHorizModule}{30mm}
\setlength{\TPVertModule}{\TPHorizModule}
\textblockorigin{2mm}{0.65\paperheight}
\setlength{\parindent}{0pt}

\newcommand{\tabitem}{~~\llap{\textbullet}~~}

%--------------------BEGIN DOCUMENT----------------------
\begin{document}

%WATERMARK TEST [**not part of a CV**]---------------
%\font\wm=''Baskerville:color=787878'' at 8pt
%\font\wmweb=''Baskerville:color=FF1493'' at 8pt
%{\wm 
%	\begin{textblock}{1}(0,0)
%		\rotatebox{-90}{\parbox{500mm}{
%			Typeset by Alessandro Plasmati with \XeTeX\  \today\ for 
%			{\wmweb \href{http://www.aleplasmati.comuv.com}{aleplasmati.comuv.com}}
%		}
%	}
%	\end{textblock}
%}

\pagestyle{empty} % non-numbered pages

\font\fb=''[cmr10]'' %for use with \LaTeX command

%--------------------TITLE-------------
\par{\centering
		{\Huge Alexandra Senderovich
	}\bigskip\par}

%--------------------SECTIONS-----------------------------------
%Section: Personal Data
%\section{Личная информация}

\begin{center}
\begin{tabular}{c c c}
     Phone number: +7 915 358 21 26 & E-mail: \href{mailto:alexandrasenderovich@gmail.com}{AlexandraSenderovich@gmail.com} & Github: \href{https://github.com/WhiteTeaDragon}{WhiteTeaDragon}
\end{tabular}
\end{center}

%Section: Work Experience at the top
%Section: Education
\section{Education}
\begin{tabular}{rl}	
2018 -- 2022 & \textbf{National Research University Higher School of Economics}, Faculty of Computer Science, \\ & Bachelor’s Programme “Applied Mathematics and Information Science”\\
& GPA: 9.67 / 10 | Cumulative rating: 1 / 225 \\
& \textbf{Relevant courses:}\\
& \tabitem Machine Learning \\
& \tabitem Deep Learning, Deep Learning in Audio \\
& \tabitem Bayesian Methods for Machine Learning \\
& \tabitem C++, Python programming \\
& \tabitem Algorithms and data structures \\
& \tabitem Computer architecture and operating systems \\
& \tabitem Distributed Systems \\
& \\
2021 & \textbf{2-week School ``Fundamentals of Bioinformatics and Mathematical Biology''} \\
& at Education Center ``Sirius'' for talented students \\
& \textbf{Courses:}\\
& \tabitem Algorithms in Bioinformatics \\
& \tabitem Protein Structure \\
& \tabitem Molecular Biology \\
& \tabitem Organic Chemistry \\
& \\
2019 & \textbf{Summer internship “Fundamentals of Computer Vision and Machine Learning”} \\
& by Associate Professor, PhD A. Konushin, grade: 10 points out of 10 \\
& \tabitem Implemented the calculation of HOG descriptors \\
& \tabitem Built models and trained artificial neural networks using scikit-learn and keras 
\end{tabular}

\section{Work Experience}
\begin{tabular}{rl}
\hspace{0.9cm} 2021 & \textbf{Summer@EPFL, Switzerland, research internship.} The project topic: \href{https://www.epfl.ch/labs/sacs/completed-student-projects/}{``D-Cliques Construction''}, \\
& research conducted at SACS (Scalable Computing Systems Lab) under supervision of Postdoc Erick Lavoie\\
& \tabitem Worked with decentralized machine learning \\
& \tabitem Developed distributed algorithms for building a communication topology and\\
& methodology for comparison of these algorithms\\
%& \tabitem Разрабатывала методологию сравнения таких алгоритмов
\end{tabular}

\section{Achievements}
\begin{tabular}{rl}
\hspace{0.9cm} 2021 & First place at \textbf{Bachelor Student Research Paper Competition} held by National Research University \\
&Higher School of Economics \\
&\\
2020, 2021 & First-degree diploma at \textbf{``Vysshaya Liga''} Olympiad in Applied Mathematics and Informatics \\
&\\
\hspace{0.9cm} 2019 & Second place with the team “Granb” at \textbf{Hack.Moscow v3.0} hackathon \\
& \tabitem Developed a parser for websites using Python's BeautifulSoup library \\
& \tabitem Implemented a search for songs using last.fm API \\
\end{tabular}

\section{Projects}
\begin{longtable}{rl}
\hspace{0.9cm} 2021 & \textbf{Group research project} ``Stable Neural Network Training Algorithm Based on SVD of Convolutional \\
&  Layers'', supervisor -- Associate Professor, PhD Maxim Rakhuba \\
& \tabitem Implemented a new compressed convolutional layer \\
& \tabitem Proved a theorem about singular values of a convolutional layer in case of multidimensional images \\
& \\
\hspace{0.9cm} 2020 & \textbf{Group software project} ``The development of a system for generating 3D-faces'' \\
& \tabitem Worked with 3D computer graphics \\
& \tabitem Implemented one out of two steps of an algorithm for example-based facial rigging\\
&\\
\hspace{0.9cm} 2020 & Participated in \textbf{The School of Future CTO} \\
& \tabitem Worked on a service for videocalls \\
& \tabitem Implemented ORM and Rest Api in Go for database interaction \\
\end{longtable}

%Section: Scholarships and additional info
\section{Scholarships}
\begin{tabular}{rl}
2018 -- current & Grant of the President of the Russian Federation for talented students \\
&\\
2022, January-June & Travel Grant and Scholarship for the winner of Bachelor Student Research Paper Competition\\
&\\
2018 -- 2020 & Moscow Government scholarship for distinguished achievements in education \\
\end{tabular}

\section{Skills}
\begin{tabular}{ll|}
Python:& \grade{5} \\
{\fb \LaTeX}: & \grade{5} \\
C/C++: & \grade{4} \\
Java: & \grade{3} \\
SQL: & \grade{3} \\
Go: & \grade{2} \\
\end{tabular}
\begin{tabular}{ll}
 \textbf{Languages}:& Russian (Native), English (C1; IELTS: 8.0 out of 9, obtained in 2020), \\ &German (A2)
\end{tabular}

\section{Scientific interests}
Machine Learning, Computer Vision, Audio Processing, Algorithms, Bioinformatics, Linguistics\\

\end{document}
