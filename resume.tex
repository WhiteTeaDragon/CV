\documentclass[a4paper,10pt]{article}

%A Few Useful Packages
\usepackage[russian, english]{babel}
\usepackage[utf8]{inputenc}
\usepackage[T2A]{fontenc}
\usepackage{marvosym}
%\usepackage{fontspec} 					%for loading fonts
\usepackage{xunicode,xltxtra,url,parskip} 	%other packages for formatting
\RequirePackage{color,graphicx}
%\usepackage[usenames,dvipsnames]{xcolor}
\usepackage[big]{layaureo} 				%better formatting of the A4 page
% an alternative to Layaureo can be ** \usepackage{fullpage} **
\usepackage{supertabular} 				%for Grades
\usepackage{titlesec}					%custom \section
\usepackage{indentfirst}
\usepackage{amsmath}
\usepackage{amssymb}
\usepackage{enumerate}
\usepackage[cm]{fullpage}
\usepackage{pgfplots}

%Setup hyperref package, and colours for links
\usepackage{hyperref}
\definecolor{linkcolour}{rgb}{0,0.2,0.6}
\hypersetup{colorlinks,breaklinks,urlcolor=linkcolour, linkcolor=linkcolour}

%FONTS
\defaultfontfeatures{Mapping=tex-text}
%\setmainfont[SmallCapsFont = Fontin SmallCaps]{Fontin}
%%% modified for Karol Kozioł for ShareLaTeX use
\setmainfont[
SmallCapsFont = Fontin-Italic.otf,
BoldFont = Fontin-Bold.otf,
ItalicFont = Fontin-Italic.otf
]
{Fontin-Regular.otf}
%%%

%CV Sections inspired by: 
%http://stefano.italians.nl/archives/26
\titleformat{\section}{\Large\scshape\raggedright}{}{0em}{}[\titlerule]
\titlespacing{\section}{0pt}{3pt}{3pt}
%Tweak a bit the top margin
%\addtolength{\voffset}{-1.3cm}

%Italian hyphenation for the word: ''corporations''
\hyphenation{im-pre-se}

%-------------WATERMARK TEST [**not part of a CV**]---------------
\usepackage[absolute]{textpos}

\setlength{\TPHorizModule}{30mm}
\setlength{\TPVertModule}{\TPHorizModule}
\textblockorigin{2mm}{0.65\paperheight}
\setlength{\parindent}{0pt}

%--------------------BEGIN DOCUMENT----------------------
\begin{document}

%WATERMARK TEST [**not part of a CV**]---------------
%\font\wm=''Baskerville:color=787878'' at 8pt
%\font\wmweb=''Baskerville:color=FF1493'' at 8pt
%{\wm 
%	\begin{textblock}{1}(0,0)
%		\rotatebox{-90}{\parbox{500mm}{
%			Typeset by Alessandro Plasmati with \XeTeX\  \today\ for 
%			{\wmweb \href{http://www.aleplasmati.comuv.com}{aleplasmati.comuv.com}}
%		}
%	}
%	\end{textblock}
%}

\pagestyle{empty} % non-numbered pages

\font\fb=''[cmr10]'' %for use with \LaTeX command

%--------------------TITLE-------------
\par{\centering
		{\Huge Александра \textsc{Сендерович}
	}\bigskip\par}

%--------------------SECTIONS-----------------------------------
%Section: Personal Data
\section{Личная информация}

\begin{tabular}{rl}
    \textsc{Телефон:}     & 8 915 358 21 26\\
    \textsc{email:}     & \href{mailto:AlexSend57@gmail.com}{AlexSend57@gmail.com}
\end{tabular}

%Section: Work Experience at the top
%Section: Education
\section{Образование}
\begin{tabular}{rl}	
2018 -- наст. время & \textbf{НИУ ВШЭ}, \textsc{факультет компьютерных наук}, \\ & образовательная программа ``Прикладная математика и информатика''\\
& GPA: 9.57 / 10 | Текущий рейтинг: 1 / 228 | Перцентиль: 0 \\
&\\
2019 & \textbf{Летняя практика} ``Основы компьютерного зрения и машинного обучения'' \\
& у доцента, к.ф.-м.н. А. Конушина пройдена на 10 баллов из 10 \\
&\\

2016 -- 2018 & \textbf{Гимназия №1543}, \textsc{математический класс}, \\
& окончила с золотой медалью \\
&\\

2015, 2016, 2017 & \textbf{Летняя компьютерная школа}\\
&\\
\end{tabular}

\section{Деятельность}
\begin{tabular}{rl}
\hspace{2.1cm} 2019 & Заняла второе место в составе команды ``Granb'' на хакатоне \textbf{Hack.Moscow v3.0} \\
\end{tabular}

%Section: Scholarships and additional info
\section{Стипендии}
\begin{tabular}{rl}
2018 -- наст. время & Грант Президента Российской Федерации за призёрство \\
& на Всероссийской олимпиаде школьников по русскому языку 2016-2017, 2017-2018 \\
2018 -- наст. время & Являюсь именным стипендиатом правительства Москвы \\
\end{tabular}

%Section: Languages
\section{Языки}
\begin{tabular}{rl}
 \textsc{Русский:}&родной\\
\textsc{Английский:}&Upper-Intermediate\\
\end{tabular}

\section{Компьютерные навыки}
\begin{tabular}{rl}
Высокий уровень владения:& C, C++, Python, {\fb \LaTeX}
\end{tabular}

\section{Интересы}
Программирование, алгоритмы, компьютерное зрение, лингвистика\\

\end{document}
