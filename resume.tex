\documentclass[a4paper,10pt]{article}

%A Few Useful Packages
\usepackage[russian, english]{babel}
\usepackage[utf8]{inputenc}
\usepackage[T2A]{fontenc}
\usepackage{marvosym}
%\usepackage{fontspec} 					%for loading fonts
\usepackage{xunicode,xltxtra,url,parskip} 	%other packages for formatting
\RequirePackage{color,graphicx}
%\usepackage[usenames,dvipsnames]{xcolor}
\usepackage[big]{layaureo} 				%better formatting of the A4 page
% an alternative to Layaureo can be ** \usepackage{fullpage} **
\usepackage{supertabular} 				%for Grades
\usepackage{titlesec}					%custom \section
\usepackage{indentfirst}
\usepackage{amsmath}
\usepackage{amssymb}
\usepackage{enumerate}
\usepackage[cm]{fullpage}
\usepackage{pgfplots}
\usepackage{array}
\usepackage{tabularx}
\usepackage{tikz}
\usepackage{multirow}

\newcommand{\grade}[1]{%
\begin{tikzpicture}
\clip (1em-.3em,-.3em) rectangle (5em +.5em ,.3em);
\begin{scope}
\clip (1em-.3em,-.3em) rectangle (#1em +.5em ,.3em);
\foreach \x in {1,2,...,5}{
 \path[fill=black] (\x em,0) circle (.25em);
}
\end{scope}
\foreach \x in {1,2,...,5}{
 \draw (\x em,0) circle (.25em);
}
\end{tikzpicture}%
}

%Setup hyperref package, and colours for links
\usepackage{hyperref}
\definecolor{linkcolour}{rgb}{0,0.2,0.6}
\hypersetup{colorlinks,breaklinks,urlcolor=linkcolour, linkcolor=linkcolour}

%FONTS
\defaultfontfeatures{Mapping=tex-text}
%\setmainfont[SmallCapsFont = Fontin SmallCaps]{Fontin}
%%% modified for Karol Kozioł for ShareLaTeX use
\setmainfont[
SmallCapsFont = Fontin-Italic.otf,
BoldFont = Fontin-Bold.otf,
ItalicFont = Fontin-Italic.otf
]
{Fontin-Regular.otf}
%%%

%CV Sections inspired by: 
%http://stefano.italians.nl/archives/26
\titleformat{\section}{\Large}{}{0em}{}[\titlerule]
\titlespacing{\section}{0pt}{3pt}{3pt}
%Tweak a bit the top margin
%\addtolength{\voffset}{-1.3cm}

%Italian hyphenation for the word: ''corporations''
\hyphenation{im-pre-se}

%-------------WATERMARK TEST [**not part of a CV**]---------------
\usepackage[absolute]{textpos}

\setlength{\TPHorizModule}{30mm}
\setlength{\TPVertModule}{\TPHorizModule}
\textblockorigin{2mm}{0.65\paperheight}
\setlength{\parindent}{0pt}

\newcommand{\tabitem}{~~\llap{\textbullet}~~}

%--------------------BEGIN DOCUMENT----------------------
\begin{document}

%WATERMARK TEST [**not part of a CV**]---------------
%\font\wm=''Baskerville:color=787878'' at 8pt
%\font\wmweb=''Baskerville:color=FF1493'' at 8pt
%{\wm 
%	\begin{textblock}{1}(0,0)
%		\rotatebox{-90}{\parbox{500mm}{
%			Typeset by Alessandro Plasmati with \XeTeX\  \today\ for 
%			{\wmweb \href{http://www.aleplasmati.comuv.com}{aleplasmati.comuv.com}}
%		}
%	}
%	\end{textblock}
%}

\pagestyle{empty} % non-numbered pages

\font\fb=''[cmr10]'' %for use with \LaTeX command

%--------------------TITLE-------------
\par{\centering
		{\Huge Александра Сендерович
	}\bigskip\par}

%--------------------SECTIONS-----------------------------------
%Section: Personal Data
%\section{Личная информация}

\begin{center}
\begin{tabular}{c c c}
     Телефон: 8 915 358 21 26 & E-mail: \href{mailto:alexandrasenderovich@gmail.com}{AlexandraSenderovich@gmail.com} & Github: \href{https://github.com/WhiteTeaDragon}{WhiteTeaDragon}
\end{tabular}
\end{center}

%Section: Work Experience at the top
%Section: Education
\section{Образование}
\begin{tabular}{rl}	
2018 -- 2022 & \textbf{НИУ ВШЭ}, факультет компьютерных наук, \\ & образовательная программа ``Прикладная математика и информатика''\\
& GPA: 9.57 / 10 | Текущий рейтинг: 1 / 228 \\
& \textbf{Релевантные курсы:}\\
& \tabitem Алгоритмы и структуры данных \\
& \tabitem Изучение языков Python, C++ \\
& \tabitem Архитектура компьютеров и операционные системы \\
& \\
2019 & \textbf{Летняя практика} ``Основы компьютерного зрения и машинного обучения'' \\
& у доцента, к.ф.-м.н. А. Конушина пройдена на 10 баллов из 10 \\
& \tabitem Реализовала извлечение признаков HOG из картинки \\
& \tabitem Использовала библиотеки scikit-learn и keras для создания моделей и тренировки нейросетей 
\end{tabular}

\section{Достижения}
\begin{tabular}{rl}
\hspace{0.9cm} 2020 & Диплом 1 степени олимпиады \textbf{Высшая лига} по профилю ``Прикладная математика и информатика'' \\
&\\
\hspace{0.9cm} 2019 & Второе место в составе команды ``Granb'' на хакатоне \textbf{Hack.Moscow v3.0} \\
& \tabitem Написала парсер для сайтов с помощью библиотеки BeautifulSoup \\
& \tabitem Работала с API сайта last.fm \\
\end{tabular}

\section{Проекты}
\begin{tabular}{rl}
\hspace{0.9cm} 2020 & \textbf{Групповой программный проект} ``Разработка системы генерации лиц трёхмерных изображений'' \\
& \tabitem Работала с трёхмерной компьютерной графикой \\
& \tabitem Реализовала один из двух шагов алгоритма переноса эмоций с одного лица на другое\\
&\\
\hspace{0.9cm} 2020 & Участвовала в \textbf{Школе будущих CTO} \\
& \tabitem Участвовала в создании сервиса видеозвонков \\
& \tabitem Написала ORM и Rest Api для взаимодействия с базами данных на Go \\
\end{tabular}

%Section: Scholarships and additional info
\section{Стипендии}
\begin{tabular}{rl}
2018 -- наст. время & \textbf{Грант Президента Российской Федерации} за призёрство \\
& на Всероссийской олимпиаде школьников по русскому языку 2016-2017, 2017-2018 \\
&\\
2018 -- наст. время & \textbf{Именная стипендия правительства Москвы} \\
\end{tabular}

\section{Навыки}
\begin{tabular}{ll|}
C/C++: & \grade{4} \\
Python:& \grade{3} \\
{\fb \LaTeX}: & \grade{3} \\
Go: & \grade{2} \\
\end{tabular}
\begin{tabular}{l}
 \textbf{Языки}: русский (родной), английский (Upper-Intermediate)
\end{tabular}

\section{Научные интересы}
Алгоритмы, компьютерное зрение, биоинформатика, лингвистика\\

\end{document}
